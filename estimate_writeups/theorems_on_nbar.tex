%% LyX 2.3.6.2 created this file.  For more info, see http://www.lyx.org/.
%% Do not edit unless you really know what you are doing.
\documentclass[english]{article}
\usepackage[latin9]{inputenc}
\usepackage{geometry}
\geometry{verbose,tmargin=2cm,bmargin=2cm,lmargin=2cm,rmargin=2cm,headheight=2cm,headsep=2cm}
\usepackage{amsmath}
\usepackage{amssymb}
\PassOptionsToPackage{normalem}{ulem}
\usepackage{ulem}
\usepackage{babel}
\begin{document}
\title{Theorems on $\bar{n}$ in AL Varying N}
\maketitle

\section{Setup}

Recall that if valuations are independent of $N$, the following upper
bound holds:

\[
F_{n:n}(v)\leq\sum_{m=n+1}^{\bar{n}}\frac{n}{(m-1)m}F_{m-1:m}(v)+\frac{n}{\bar{n}}F_{\bar{n}-1:\bar{n}}(v)
\]
And the following lower bound holds:\\
\[
F_{n:n}(v)\geq\sum_{m=n+1}^{\bar{n}}\frac{n}{(m-1)m}F_{m-1:m}(v)+\frac{n}{\bar{n}}\left(\phi_{\bar{n}-1:\bar{n}}\left(F_{\bar{n}-1:\bar{n}}(v)\right)\right)^{\bar{n}}
\]
\\


\section{Theorems}

\subsection{Theorem 1}

Let $S=\{n+1,n+2,...,\bar{n}\}$ be the set of all possible upper
limits of summation. The upper bound generated with $\max(S)=\bar{n}$
is the tightest upper bound for $F_{n:n,s}$ from all of $s\in S$.\\
\\
\uline{Proof}:\\
Define $\bar{n}_{2},\bar{n}_{1}\in\mathbb{N}$ and let $n<\bar{n}_{1}=\bar{n}_{2}-1\leq\bar{n}$.
Then take the difference of these upper bounds, 
\begin{align*}
F_{n:n,\bar{n}_{1}}(v)-F_{n:n,\bar{n}_{2}}(v) & =\frac{n}{\bar{n}_{1}}F_{\bar{n}_{1}-1:\bar{n}_{1}}(v)-\frac{n}{\bar{n}_{2}}F_{\bar{n}_{2}-1:\bar{n}_{2}}(v)-\frac{n}{\bar{n}_{2}(\bar{n}_{2}-1)}F_{\bar{n}_{2}-1:\bar{n}_{2}}(v)\\
 & =\frac{n}{\bar{n}_{1}}F_{\bar{n}_{1}-1:\bar{n}_{1}}(v)-F_{\bar{n}_{2}-1:\bar{n}_{2}}(v)*\left(\frac{n+n\left(\bar{n}_{2}-1\right)}{\bar{n}_{2}\left(\bar{n}_{2}-1\right)}\right)\\
 & =\frac{n}{\bar{n}_{1}}F_{\bar{n}_{1}-1:\bar{n}_{1}}(v)-\frac{n}{\bar{n}_{2}-1}F_{\bar{n}_{2}-1:\bar{n}_{2}}(v)\\
 & =\frac{n}{\bar{n}_{1}}\left(F_{\bar{n}_{1}-1:\bar{n}_{1}}(v)-F_{\bar{n}_{2}-1:\bar{n}_{2}}(v)\right)
\end{align*}
Recall that the CDF of the second highest order statistic is $F_{k-1:k}(v)=kF(v)^{k-1}-(k-1)F(v)^{k}$.
So, \\
\begin{align*}
F_{\bar{n}_{1}-1:\bar{n}_{1}}(v)-F_{\bar{n}_{2}-1:\bar{n}_{2}}(v) & =\bar{n}_{1}F^{\bar{n}_{1}-1}(v)-\left(\bar{n}_{1}-1\right)F^{\bar{n}_{1}}(v)-\left[(\bar{n}_{1}+1)F^{\bar{n}_{1}}(v)-\bar{n}_{1}F^{\bar{n}_{1}+1}(v)\right]\\
 & =\bar{n}_{1}F^{\bar{n}_{1}-1}(v)-2\bar{n}_{1}F^{\bar{n}_{1}}(v)+\bar{n}_{1}F^{\bar{n}_{1}+1}(v)\\
 & =\bar{n}_{1}\left[F^{\bar{n}_{1}-1}(v)-2F^{\bar{n}_{1}}(v)+F^{\bar{n}_{1}+1}(v)\right]\\
 & =\bar{n}_{1}F(v)^{\bar{n}_{1}-1}\left[1-2F(v)+F(v)^{2}\right]\\
 & =\bar{n}_{1}F(v)^{\bar{n}_{1}-1}\left(1-F(v)\right)(1-F(v))\\
 & \geq0
\end{align*}
So $F_{n:n,\bar{n}_{1}}(v)\geq F_{n:n,\bar{n}_{2}}(v)$ for all $v$.
It follows by an inductive argument that $F_{n:n,\bar{n}}(v)$ is
indeed the lowest upper bound.\\
\\


\subsection{Theorem 2}

Let $S=\{n+1,n+2,...,\bar{n}\}$ be the set of all possible upper
limits of summation. If valuations are $i.i.d.$, the lower bounds
for $F_{n:n,s}$ from $s\in S$ are all equally tight.\\
\\
\uline{Proof}:\\
Define $\bar{n}_{2},\bar{n}_{1}\in\mathbb{N}$ and let $n<\bar{n}_{1}=\bar{n}_{2}-1\leq\bar{n}$.
Then take the difference of these upper bounds, 
\begin{align*}
F_{n:n,\bar{n}_{2}}(v)-F_{n:n,\bar{n}_{1}}(v) & =\frac{n}{\bar{n}_{2}(\bar{n}_{2}-1)}F_{\bar{n}_{2}-1:\bar{n}_{2}}(v)+\frac{n}{\bar{n}_{2}}\left[\phi_{\bar{n}_{2}-1:\bar{n}_{2}}\left(F_{\bar{n}_{2}-1:\bar{n}_{2}}(v)\right)\right]^{\bar{n}_{2}}-\frac{n}{\bar{n}_{1}}\left[\phi_{\bar{n}_{1}-1:\bar{n}_{1}}\left(F_{\bar{n}_{1}-1:\bar{n}_{1}}(v)\right)\right]^{\bar{n}_{1}}\\
 & =\frac{n}{\bar{n}_{2}(\bar{n}_{2}-1)}\left\{ F_{\bar{n}_{2}-1:\bar{n}_{2}}(v)+\left(\bar{n}_{2}-1\right)\left[\phi_{\bar{n}_{2}-1:\bar{n}_{2}}\left(F_{\bar{n}_{2}-1:\bar{n}_{2}}(v)\right)\right]^{\bar{n}_{2}}-\left(\bar{n}_{2}\right)\left[\phi_{\bar{n}_{1}-1:\bar{n}_{1}}\left(F_{\bar{n}_{1}-1:\bar{n}_{1}}(v)\right)\right]^{\bar{n}_{1}}\right\} 
\end{align*}
Since valuations are $i.i.d.$, $\phi_{k-1:k}\left(F_{k-1:k}(v)\right)=F(v)$,
so we have, \\
\begin{align*}
F_{n:n,\bar{n}_{2}}(v)-F_{n:n,\bar{n}_{1}}(v) & =\frac{n}{\bar{n}_{2}(\bar{n}_{2}-1)}\left\{ F_{\bar{n}_{2}-1:\bar{n}_{2}}(v)+\bar{n}_{1}F(v)^{\bar{n}_{2}}-\bar{n}_{2}F(v)^{\bar{n}_{1}}\right\} \\
 & =\frac{n}{\bar{n}_{2}(\bar{n}_{2}-1)}\left\{ F_{\bar{n}_{2}-1:\bar{n}_{2}}(v)-\left[\bar{n}_{2}F(v)^{\bar{n}_{1}}-\bar{n}_{1}F(v)^{\bar{n}_{2}}\right]\right\}  & \text{(i)}\\
 & =\frac{n}{\bar{n}_{2}(\bar{n}_{2}-1)}\left\{ F_{\bar{n}_{2}-1:\bar{n}_{2}}(v)-F_{\bar{n}_{2}-1:\bar{n}_{2}}(v)\right\} \\
 & =0
\end{align*}
(i): The CDF of the second highest order statistic is $F_{k-1:k}(v)=kF(v)^{k-1}-(k-1)F(v)^{k}$.\\
\\
So we have $F_{n:n,\bar{n}_{2}}(v)=F_{n:n,\bar{n}_{1}}(v)$. Then,
again following an inductive argument, the bounds for $F_{n:n}$ from
$S$ are all equally tight.\\
\\


\subsection{Theorem 3}

Let $S=\{n+1,n+2,...,\bar{n}\}$ be the set of all possible upper
limits of summation. If valuations are common, the lower bounds for
$F_{n:n,s}$ from $s\in S$ are increasing in in $s$.\\
\\
\uline{Proof}:\\
Define $\bar{n}_{2},\bar{n}_{1}\in\mathbb{N}$ and let $n<\bar{n}_{1}=\bar{n}_{2}-1\leq\bar{n}$.
If valuations are common, $F_{\bar{n}_{2}-1:\bar{n}_{2}}(v)=F_{\bar{n}_{1}-1:\bar{n}_{1}}(v)\equiv H$.
So write, \\
\begin{align*}
F_{n:n,\bar{n}_{2}}(v)-F_{n:n,\bar{n}_{1}}(v) & =\frac{n}{\bar{n}_{2}(\bar{n}_{2}-1)}\left\{ H+\bar{n}_{1}\left[\phi_{\bar{n}_{2}-1:\bar{n}_{2}}\left(H\right)\right]^{\bar{n}_{2}}-\bar{n}_{2}\left[\phi_{\bar{n}_{1}-1:\bar{n}_{1}}\left(H\right)\right]^{\bar{n}_{1}}\right\} \\
 & =\frac{n}{\bar{n}_{2}(\bar{n}_{2}-1)}\left\{ H+\bar{n}_{1}\left[\phi_{\bar{n}_{1}:\bar{n}_{1}+1}\left(H\right)\right]^{\bar{n}_{1}+1}-\left(\bar{n}_{1}+1\right)\left[\phi_{\bar{n}_{1}-1:\bar{n}_{1}}\left(H\right)\right]^{\bar{n}_{1}}\right\} 
\end{align*}
So we want to show that $H+\bar{n}_{1}\left[\phi_{\bar{n}_{1}:\bar{n}_{1}+1}\left(H\right)\right]^{\bar{n}_{1}+1}-\left(\bar{n}_{1}+1\right)\left[\phi_{\bar{n}_{1}-1:\bar{n}_{1}}\left(H\right)\right]^{\bar{n}_{1}}\geq0$.\\
\uline{Lemma}: $\phi_{n:n+1}\left(H\right)\geq\phi_{n-1:n}\left(H\right)$
for all $n\geq2$ and $H\in[0,1]$. \\
\uline{Proof}: For reduced notation let $\phi\equiv\phi_{n-1:n}$
in this proof. Recall that $H=n\phi^{n-1}-(n-1)\phi^{n}$. Now differentiate
this with respect to $n$. To do so, we need to use the identity $\left(\left(\alpha_{n}\right)^{\beta(n)}\right)'=\alpha(n)^{\beta(n)}\beta'(n)\log\alpha(n)+\alpha(n)^{\beta(n)-1}$.
So, 
\begin{align*}
\left(n\phi^{n-1}\right)'_{n} & =\phi^{n-1}+n\left[\phi^{n-1}\log\phi+\phi^{n-2}(n-1)\phi'\right]\\
\left((n-1)\phi^{n}\right)'_{n} & =\phi^{n}+(n-1)\left[\phi^{n}\log\phi+\phi^{n-1}n\phi'\right]
\end{align*}
Then it follows that,\\
\[
\phi^{n-1}+n\left[\phi^{n-1}\log\phi+\phi^{n-2}(n-1)\phi'\right]=\phi^{n}+(n-1)\left[\phi^{n}\log\phi+\phi^{n-1}n\phi'\right]
\]
Rearranging, 
\[
\phi'=\frac{\phi-\phi^{2}+n\phi\log\phi-(n-1)\phi^{2}\log\phi}{n(n-1)(\phi-1)}
\]
Recall that $\phi\in[0,1]$, so to show this derivative is $\geq0$,
we only need to show that the numerator $\leq0$. We can write, 
\begin{align*}
1-\phi+n\log\phi-(n-1)\phi\log\phi & =1-\phi+\left[n-(n-1)\phi\right]\log\phi\\
 & \leq1-\phi+\left[n-(n-1)\phi\right]\left(\phi-1\right) & \text{(i)}\\
 & =\left(1-\phi\right)^{2}\left[1-n\right]\\
 & \leq0
\end{align*}
(i): This follows from the Taylor expansion centered at $1$: $\log x=\left(x-1\right)-\frac{(x-1)^{2}}{2}+\frac{(x-1)^{3}}{3}-\frac{(x-1)^{4}}{4}+...$.
So $\log(x)\leq x-1$ for all $x\in[0,1]$. $\blacksquare$\\
\\
Now, again recall that $H=k\phi_{k-1:k}^{k-1}-(k-1)\phi_{k-1:k}^{k}$.
Substitute $H$ where $k=\bar{n}_{1}+1$, then
\begin{align*}
 & H+\bar{n}_{1}\left[\phi_{\bar{n}_{1}:\bar{n}_{1}+1}\left(H\right)\right]^{\bar{n}_{1}+1}-\left(\bar{n}_{1}+1\right)\left[\phi_{\bar{n}_{1}-1:\bar{n}_{1}}\left(H\right)\right]^{\bar{n}_{1}}\\
= & \left(\bar{n}_{1}+1\right)\left[\phi_{\bar{n}_{1}:\bar{n}_{1}+1}\left(H\right)\right]^{\bar{n}_{1}}-\left(\bar{n}_{1}+1\right)\left[\phi_{\bar{n}_{1}-1:\bar{n}_{1}}\left(H\right)\right]^{\bar{n}_{1}}\\
= & \left(\bar{n}_{1}+1\right)\left[\phi_{\bar{n}_{1}:\bar{n}_{1}+1}\left(H\right)^{\bar{n}_{1}}-\phi_{\bar{n}_{1}-1:\bar{n}_{1}}\left(H\right)^{\bar{n}_{1}}\right]
\end{align*}
But from the Lemma, $\phi_{\bar{n}_{1}:\bar{n}_{1}+1}\left(H\right)\geq\phi_{\bar{n}_{1}-1:\bar{n}_{1}}\left(H\right)$.
So $F_{n:n,\bar{n}_{2}}(v)-F_{n:n,\bar{n}_{1}}(v)\geq0$. It follows
by an inductive argument that the lower bounds for $F_{n:n,s}$ from
$s\in S$ are increasing in in $s$, and thus $F_{n:n,\bar{n}}(v)$
is indeed the greatest lower bound.\\
\\
\\
\\
\\
\\

\end{document}
