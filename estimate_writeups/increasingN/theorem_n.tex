%% LyX 2.3.6.2 created this file.  For more info, see http://www.lyx.org/.
%% Do not edit unless you really know what you are doing.
\documentclass[english]{article}
\usepackage[latin9]{inputenc}
\usepackage{amsmath}
\usepackage{amssymb}
\PassOptionsToPackage{normalem}{ulem}
\usepackage{ulem}
\usepackage{babel}
\begin{document}
\title{A Theorem on $n$ }
\date{January 11, 2023}
\author{Bruce Wen}
\maketitle

\section{Theorem}

From Haile Tamer (2003), define the strictly increasing differentiable
function $\phi_{i:n}(H):[0,1]\to[0,1]$ as the implicit solution to
\[
H=\frac{n!}{(n-i)!(i-1)!}\int_{0}^{\phi}s^{i-1}(1-s)^{n-i}ds
\]
Define $G_{n:n}(v)$ as the highest order statistic among a known
$n$ bidders empirical distribution function with support $[\text{\ensuremath{\underbar{v}}},\bar{v}]$.\\
Define the function, $f_{n}(v):[0,1]\to[0,1]$, 
\[
f_{n}(v)=\phi_{n:n}^{-1}\phi_{n-1:n}\left(G_{n:n}(v)\right)
\]
\\
\textbf{Theorem 1}: Suppose $n\in\mathbb{N}$ and $n>1$. Then $f_{n}(v)$
is decreasing in $n\in\mathbb{N}$, for all $\text{\ensuremath{\underbar{v}}}\leq v\leq\bar{v}$.\\
\\
\uline{Note}: Why is $f_{n}(v)$ like this? It is used in estimating
the $F_{n:n}$ (true distribution - $G$ is the observed distribution)
order statistic (theorem by Kirill Ponomarev 2023) as, 
\[
\phi_{n:n}^{-1}\phi_{n-1:n}\left(G_{n:n}(v)\right)\leq F_{n:n}(v)\leq G_{n:n}(b)
\]
\\
\\
\\


\section{Proof}

\textbf{Lemma 3}: $f_{n}(v):[0,1]\to[0,1]$, $f_{n}(v)=\phi_{n:n}^{-1}\phi_{n-1:n}\left(G_{n:n}(v)\right)$
can be rewritten as:\\
\[
t^{n}=ny^{\frac{n-1}{n}}-(n-1)y
\]
\uline{Proof}: \\
The function can be rewritten, 
\begin{align*}
\phi_{n:n}^{-1}\phi_{n-1:n}\left(G_{n:n}(v)\right) & =\phi_{n-1:n}\left(G_{n:n}(v)\right)^{n}\\
 & =\phi_{n-1:n}\left(v^{n}\right)^{n}
\end{align*}
Let $y\equiv\phi_{n:n}^{-1}\phi_{n-1:n}\left(G_{n:n}(v)\right)$ and
$t\equiv v$. Then, $y=\phi_{n-1:n}\left(t^{n}\right)^{n}$. And so,
\[
y^{\frac{1}{n}}=\phi_{n-1:n}\left(t^{n}\right)
\]
Then, 
\begin{align*}
t^{n} & =\frac{n!}{(n-2)!}\int_{0}^{y^{\frac{1}{n}}}s^{n-2}(1-s)ds\\
 & =n(n-1)\left[\frac{1}{n-1}s^{n-1}-\frac{1}{n}s^{n}\right]_{0}^{y^{\frac{1}{n}}}\\
 & =ny^{\frac{n-1}{n}}-(n-1)y
\end{align*}
\\
\textbf{Lemma 4}: Define the implicit equation, 
\[
t^{n}=ny^{\frac{n-1}{n}}-(n-1)y
\]
For any fixed $t\in[0,1]$, and $n\geq2$ integers, $y\in[0,1]$ is
decreasing in $n$.\\
\\
\\
\\
\\
\\
\textbf{\uline{REDACTED}}\\
\textbf{Lemma 1}: $G_{n+1:n+1}(v)\succsim_{\text{FOSD}}G_{n:n}(v)$
for all $n\in\mathbb{N}$ and all $v$.\\
\uline{Proof}:\\
Take a common distribution $G$ where valuations are drawn from. In
1 scenario, $n+1$ valuations are drawn from $G$, and we can define
the CDF of the highest order statistic,$G_{n+1:n+1}(v)=G(v)^{n+1}$.
In the other scenario, $n$ valuations are drawn from $G$, and the
CDF for this highest order statistic is, $G_{n:n}(v)=G(v)^{n}$. Thus,
\[
G_{n+1:n+1}(v)\leq G_{n:n}(v)
\]
for all $v$ and $n$, and so $G_{n+1:n+1}(v)\succsim_{\text{FOSD}}G_{n:n}(v)$.
$\blacksquare$\\
\\
\\
We just need to prove that $\phi_{n:n}^{-1}\phi_{n-1:n}\left(t\right)$
is decreasing in $n$ for all $t\in[0,1]$. \\
\\
First, see that $\phi_{n:n}^{-1}(x)=x^{n}$: 
\begin{align*}
H & =\frac{n!}{(n-n)!(n-1)!}\int_{0}^{\phi}s^{n-1}(1-s)^{n-n}ds\\
 & =n\int_{0}^{\phi}s^{n-1}ds\\
 & =n\frac{1}{n}\left[s^{n}\right]_{0}^{\phi}\\
 & =\phi^{n}
\end{align*}
\\
It remains to show that $\phi_{n-1:n}\left(t\right)^{n}$ is decreasing
in $n$ for all $t\in(0,1)$. \\
\\
\textbf{Lemma 2}: $\phi_{n-1:n}\left(t\right)$ is decreasing(?increasing?)
in $n$ for all $t\in(0,1)$. \\
\\

\end{document}
