%% LyX 2.3.6.2 created this file.  For more info, see http://www.lyx.org/.
%% Do not edit unless you really know what you are doing.
\documentclass[english]{article}
\usepackage[latin9]{inputenc}
\usepackage{amsmath}
\usepackage{amssymb}
\usepackage{babel}
\begin{document}
\title{A Theorem on $n$ - Updated Proof}
\author{Bruce Wen, Kirill Ponomarev}
\date{March 2, 2023}
\maketitle

\section{Theorem}

From Haile Tamer (2003), define the strictly increasing differentiable
function $\phi_{i:n}(H):[0,1]\to[0,1]$ as the implicit solution to
\[
H=\frac{n!}{(n-i)!(i-1)!}\int_{0}^{\phi}s^{i-1}(1-s)^{n-i}ds
\]
Define $G_{n:n}(t)$ as the highest order statistic among a known
$n$-bidder empirical distribution function with support $[\underline{v},\bar{v}]$.\\
Define the function, $f_{n}(t):[0,1]\to[0,1]$, 
\[
f_{n}(t)=\phi_{n:n}^{-1}\phi_{n-1:n}\left(G_{n:n}(t)\right)
\]
\\
\textbf{Theorem}: Suppose $n\in\mathbb{N}$ and $n\geq2$. Then $f_{n}(t)$
is decreasing in $n\in\mathbb{N}$, for all $0\leq t\leq1$.\\
\\


\section{Proof}

First observe that $\phi_{n:n}^{-1}(x)=x^{n}$ and that the highest
order statistic is simply raising to the $n^{th}$ power, so we can
write,
\[
f_{n}(t)=\phi_{n-1:n}\left(t^{n}\right)^{n}
\]
For brevity of notation, from now on let $\phi\equiv\phi_{n-1:n}$.\\
Consider the identity, $\left(\alpha(n)^{\beta(n)}\right)'=\alpha(n)^{\beta(n)}\beta'(n)\log\alpha(n)+\alpha(n)^{\beta(n)-1}\beta(n)\alpha'(n)$.
Use this to take the derivative with respect to $n$, 
\begin{align*}
\left(f_{n}(t)\right)_{n}^{'} & =\phi\left(t^{n}\right)^{n}\log\phi\left(t^{n}\right)+\phi\left(t^{n}\right)^{n-1}n\left(\phi\left(t^{n}\right)\right)_{n}' & (*)
\end{align*}
\\
First compute $\left(\phi\left(t^{n}\right)\right)_{n}'$ in $(*)$,
which can be written, 
\begin{align*}
\left(\phi\left(t^{n},n\right)\right)_{n}^{'} & =\underset{\text{Term 1}}{\underbrace{\phi'_{t}\left(t^{n},n\right)}}\cdot t^{n}\log t+\underset{\text{Term 2}}{\underbrace{\phi'_{n}\left(t^{n},n\right)}}
\end{align*}
\textbf{Term 1} can be computed as follows. For some $H\in[0,1]$,
observe that $\phi_{n-1:n}(H)$ is the implicit solution to the following:
\begin{align*}
H & =\frac{n!}{(n-n+1)!(n-1-1)!}\int_{0}^{\phi}s^{n-1-1}(1-s)^{n-n+1}ds\\
 & =n(n-1)\left[\frac{1}{n-1}s^{n-1}-\frac{1}{n}s^{n}\right]_{0}^{\phi}\\
 & =n\phi^{n-1}-(n-1)\phi^{n}
\end{align*}
And so, the inverse can be written as $\phi^{-1}(H)=nH^{n-1}-(n-1)H^{n}$.
Differentiating this,
\begin{align*}
\left(\phi^{-1}\right)^{'}(H) & =n(n-1)H^{n-2}-n(n-1)H^{n-1}\\
 & =n(n-1)H^{n-2}(1-H)
\end{align*}
And so by the inverse function theorem, we have
\[
\phi'_{t}\left(t,n\right)=\frac{1}{n(n-1)\phi(t,n)^{n-2}\left[1-\phi\left(t,n\right)\right]}
\]
We will now compute \textbf{Term 2}. \\
Again use the identity $\left(\alpha(n)^{\beta(n)}\right)'=\alpha(n)^{\beta(n)}\beta'(n)\log\alpha(n)+\alpha(n)^{\beta(n)-1}\beta(n)\alpha'(n)$
to differentiate $t=n\phi^{n-1}\left(t,n\right)-(n-1)\phi^{n}\left(t,n\right)$
implicitly, we get (abbreviating the notation $\phi\equiv\phi(t^{n},n)$),
\begin{align*}
\phi'_{n}\left(t^{n},n\right) & =\frac{\phi^{n}-\phi^{n-1}-t^{n}\log\phi}{n(n-1)\phi^{n-2}(1-\phi)}
\end{align*}
This completes Term 2.\\
Adding them, 
\[
\left(\phi\left(t^{n},n\right)\right)_{n}^{'}=\frac{t^{n}\log\frac{t}{\phi}+\phi^{n}-\phi^{n-1}}{n(n-1)\phi^{n-2}(1-\phi)}
\]
Now note that 
\begin{align*}
\left(\phi\left(t^{n},n\right)\right)_{n}^{'} & \geq0\;\;\forall n\geq1 & (**)
\end{align*}
To see this, first observe that the denominator is positive, so we
only need to consider $t^{n}\log\frac{t}{\phi}$ versus $\phi^{n}-\phi^{n-1}$.
Recall that $t^{n}=n\phi^{n-1}-(n-1)\phi^{n}$, so 
\begin{align*}
\phi^{n-1}-\phi^{n} & =\frac{t^{n}-\phi^{n}}{n} & (***)
\end{align*}
, so we are comparing $\log\frac{t}{\phi}$ versus $\frac{1}{n}\left(1-\frac{1}{\left(\frac{t}{\phi}\right)^{n}}\right)$.
But $\log(x)\geq\frac{1}{n}\left(1-\frac{1}{x^{n}}\right)$ for all
$x\geq0$, $n\geq1$.\\
\\
Plugging $\left(\phi\left(t^{n},n\right)\right)_{n}^{'}$ back into
$(*)$, we have, \\
\begin{align*}
\left(f_{n}(t)\right)_{n}' & =\phi{}^{n}\log\phi+\phi{}^{n-1}n\left(\phi\left(t^{n},n\right)\right)_{n}'\\
 & =\phi{}^{n}\log\phi+\phi{}^{n-1}\frac{t^{n}\log\frac{t}{\phi}+\phi^{n}-\phi^{n-1}}{(n-1)\phi^{n-2}(1-\phi)}\\
 & =\frac{\phi}{(n-1)(1-\phi)}\left[\left(\phi^{n-1}-\phi^{n}\right)\left((n-1)\log\phi-1\right)+t^{n}\log\left(\frac{t}{\phi}\right)\right]\\
 & =\frac{t^{n}\phi}{(n-1)(1-\phi)}\underset{(****)}{\underbrace{\left[\frac{1}{n}\left(1-\frac{1}{\left(\frac{t}{\phi}\right)^{n}}\right)\left((n-1)\log\phi-1\right)+\log\left(\frac{t}{\phi}\right)\right]}}
\end{align*}
We need to show that $(****)\leq0$ for all $n\geq2$. Observe that,
\[
(****)\leq0\iff\log\phi\leq\frac{1}{n-1}\cdot\frac{\frac{1}{n}\left(1-\frac{1}{\left(\frac{t}{\phi}\right)^{n}}\right)-\log\frac{t}{\phi}}{\frac{1}{n}\left(1-\frac{1}{\left(\frac{t}{\phi}\right)^{n}}\right)}
\]
Let $z\equiv\frac{t}{\phi}$. Then, 
\[
(****)\leq0\iff\log t\leq\underset{g(z,n)}{\underbrace{\log z+\frac{1}{n-1}\cdot\frac{\frac{1}{n}\left(1-\frac{1}{z^{n}}\right)-\log z}{\frac{1}{n}\left(1-\frac{1}{z^{n}}\right)}}}
\]
Now, see that $\log t$ does not depend on $n$, while in the RHS,
the following holds:
\begin{enumerate}
\item $g(z,n)$ is increasing in $n$ for a fixed $z$.
\item $g(z,n)$ is decreasing in $z$ for a fixed $n$.
\end{enumerate}
But recall that $z=z_{n}=\frac{t}{\phi_{n-1:n}\left(t^{n}\right)}$,
which is decreasing in $n$ for a fixed $t$ from $(**)$. So $g(z_{n},n)$
is increasing in $n$, and thus it suffices to check $(****)$ for
$n=2$. \\
\\
For the case of $n=2$, consider the following properties:
\begin{enumerate}
\item $\frac{z^{2}}{z^{2}-1}=\frac{1}{2}\left(\frac{1}{1-\phi}+1\right)$
by substituting $z=\frac{t}{\phi}$ and using $(***)$ for $n=2$,
$2\phi-\phi^{2}=t^{2}$.
\item $\log z=\frac{1}{2}\left(\log(2-\phi)-\log\phi\right)$ by again using
$2\phi-\phi^{2}=t^{2}$ and taking the natural log.
\end{enumerate}
Then, 

\begin{align*}
\log t-g(z_{2},2) & =\log t-\log z-1+\frac{z^{2}}{z^{2}-1}\cdot2\log z\\
 & =\log\phi-1+\frac{1}{2}\left(\frac{1}{1-\phi}+1\right)\left(\log(2-\phi)-\log\phi\right)\\
 & =\log\phi-1+\frac{\log(2-\phi)-\log\phi}{2(1-\phi)}+\frac{\log(2-\phi)-\log\phi}{2}\\
 & =\frac{\log(2-\phi)-\log\phi}{2(1-\phi)}+\frac{\log(2-\phi)+\log\phi}{2}-1\\
 & =\log(\phi)+\frac{(\phi-2)\tanh^{-1}(1-\phi)}{\phi-1}-1\\
 & \leq0\;\;\;\;\;\;\forall\phi\in(0,1)
\end{align*}
So $(****)$ is true for $n=2$, which proves that $(****)\leq0$.
\\
$\blacksquare$
\end{document}
