%% LyX 2.3.6.2 created this file.  For more info, see http://www.lyx.org/.
%% Do not edit unless you really know what you are doing.
\documentclass[english]{article}
\usepackage[latin9]{inputenc}
\usepackage{amsmath}
\usepackage{amssymb}
\PassOptionsToPackage{normalem}{ulem}
\usepackage{ulem}
\usepackage{babel}
\begin{document}
\title{A Theorem on $n$ }
\date{January 11, 2023}
\author{Bruce Wen}
\maketitle

\section{Theorem}

From Haile Tamer (2003), define the strictly increasing differentiable
function $\phi_{i:n}(H):[0,1]\to[0,1]$ as the implicit solution to
\[
H=\frac{n!}{(n-i)!(i-1)!}\int_{0}^{\phi}s^{i-1}(1-s)^{n-i}ds
\]
Define $G_{n:n}(v)$ as the highest order statistic among a known
$n$-bidder empirical distribution function with support $[\text{\ensuremath{\underbar{v}}},\bar{v}]$.\\
Define the function, $f_{n}(v):[0,1]\to[0,1]$, 
\[
f_{n}(v)=\phi_{n:n}^{-1}\phi_{n-1:n}\left(G_{n:n}(v)\right)
\]
\\
\textbf{Theorem 1}: Suppose $n\in\mathbb{N}$ and $n>1$. Then $f_{n}(v)$
is decreasing in $n\in\mathbb{N}$, for all $\text{\ensuremath{\underbar{v}}}\leq v\leq\bar{v}$.\\
\\


\section{Proof}

\textbf{Lemma 0.1}: $\phi_{n:n}^{-1}(x)=x^{n}$.\\
\uline{Proof}: See that : 
\begin{align*}
H & =\frac{n!}{(n-n)!(n-1)!}\int_{0}^{\phi}s^{n-1}(1-s)^{n-n}ds\\
 & =n\int_{0}^{\phi}s^{n-1}ds\\
 & =n\frac{1}{n}\left[s^{n}\right]_{0}^{\phi}\\
 & =\phi^{n}
\end{align*}
\textbf{}\\
\textbf{Lemma 1}: If $n,n'\in\mathbb{N}$ and $n'>n$, then $G_{n':n'}(v)\succsim_{\text{FOSD}}G_{n:n}(v)$
for all $v\in[0,1]$.\\
\uline{Proof}:\\
It suffices to show that $G_{n+1:n+1}(v)\succsim_{\text{FOSD}}G_{n:n}(v)$
for all $v$.\\
Take a common distribution $G$ where valuations are drawn from. In
1 scenario, $n+1$ valuations are drawn from $G$, and we can define
the CDF of the highest order statistic,$G_{n+1:n+1}(v)=G(v)^{n+1}$.
In the other scenario, $n$ valuations are drawn from $G$, and the
CDF for this highest order statistic is, $G_{n:n}(v)=G(v)^{n}$. Thus,
\[
G_{n+1:n+1}(v)\leq G_{n:n}(v)
\]
for all $v$ and $n$, and so $G_{n+1:n+1}(v)\succsim_{\text{FOSD}}G_{n:n}(v)$.
$\blacksquare$\\
\\
\textbf{Lemma 2}: $\phi_{n:n}^{-1}\phi_{n-1:n}\left(G_{n:n}(v)\right)$
is decreasing in $n\in\mathbb{N},n>1$ if the following holds: $\phi_{n-1:n}(H)$
is increasing in $H\in[0,1]$ for any $n\geq2$.\\
\\
\uline{Proof}:\\
We can rewrite $\phi_{n:n}^{-1}\phi_{n-1:n}\left(G_{n:n}(v)\right)$
as $\phi_{n-1:n}\left(v^{n}\right)^{n}$ by Lemma 0.1. From Lemma
1, $G_{n:n}(v)$ is decreasing in $n$ for all $v\in[0,1]$. Furthermore,
$\phi_{n-1:n}(.)$ maps to $[0,1]$, so the outer exponent is decreasing
in $n$ as well. Thus, to show $\phi_{n-1:n}\left(v^{n}\right)^{n}$
is decreasing in $n$, it suffices to show that $\phi_{n-1:n}(H)$
is increasing in $H\in[0,1]$ for any $n\geq2,n\in\mathbb{N}$. $\blacksquare$\\
\\
\textbf{Lemma 3}: $\phi_{n-1:n}(H)$ is increasing in $H\in[0,1]$
for any $n\geq2$.\\
\uline{Proof}:\\
Observe that $\phi_{n-1:n}(H)$ is the implicit solution to the following:
\begin{align*}
H & =\frac{n!}{(n-n+1)!(n-1-1)!}\int_{0}^{\phi}s^{n-1-1}(1-s)^{n-n+1}ds\\
 & =n(n-1)\left[\frac{1}{n-1}s^{n-1}-\frac{1}{n}s^{n}\right]_{0}^{\phi}\\
 & =n\phi^{n-1}-(n-1)\phi^{n}
\end{align*}
We will now show that the inverse of $\phi_{n-1:n}(.)$, $H(n,\phi)=n\phi^{n-1}-(n-1)\phi^{n}$,
is increasing in $\phi\in(0,1)$ for all $n\geq2$. The partial derivative
is, 
\begin{align*}
\frac{\partial H(n,\phi)}{\partial\phi} & =n(n-1)\phi^{n-2}-n(n-1)\phi^{n-1}\\
 & =n(n-1)\left[\phi^{n-2}-\phi^{n-1}\right]\\
 & \geq0
\end{align*}
where the last inequality comes from the fact that $\phi\in[0,1]$
and $n\geq2$. \\
Since the inverse of $\phi_{n-1:n}(.)$ is increasing, $\phi_{n-1:n}(.)$
must be increasing as well. $\blacksquare$\\
\\
The prooof of \textbf{Theorem 1} follows from \textbf{Lemmas 1,2,3}.\\
\\
\\


\section{Extensions}

\textbf{Lemma 4}: $f_{n}(v):[0,1]\to[0,1]$, $f_{n}(v)=\phi_{n:n}^{-1}\phi_{n-1:n}\left(G_{n:n}(v)\right)$
can be rewritten as:\\
\[
t^{n}=ny^{\frac{n-1}{n}}-(n-1)y
\]
\uline{Proof}: \\
The function can be rewritten, 
\begin{align*}
\phi_{n:n}^{-1}\phi_{n-1:n}\left(G_{n:n}(v)\right) & =\phi_{n-1:n}\left(G_{n:n}(v)\right)^{n}\\
 & =\phi_{n-1:n}\left(v^{n}\right)^{n}
\end{align*}
Let $y\equiv\phi_{n:n}^{-1}\phi_{n-1:n}\left(G_{n:n}(v)\right)$ and
$t\equiv v$. Then, $y=\phi_{n-1:n}\left(t^{n}\right)^{n}$. And so,
\[
y^{\frac{1}{n}}=\phi_{n-1:n}\left(t^{n}\right)
\]
Then, 
\begin{align*}
t^{n} & =\frac{n!}{(n-2)!}\int_{0}^{y^{\frac{1}{n}}}s^{n-2}(1-s)ds\\
 & =n(n-1)\left[\frac{1}{n-1}s^{n-1}-\frac{1}{n}s^{n}\right]_{0}^{y^{\frac{1}{n}}}\\
 & =ny^{\frac{n-1}{n}}-(n-1)y
\end{align*}
\\
\textbf{Theorem 2}: Define the implicit equation, 
\[
t^{n}=ny^{\frac{n-1}{n}}-(n-1)y
\]
For any fixed $t\in[0,1]$, and $n\geq2$ integers, $y\in[0,1]$ is
decreasing in $n$.\\
\\
\uline{Proof}: This follows from \textbf{Lemma 4} and \textbf{Theorem
1}.\\

\end{document}
